\input{configuration}

\title{Lecture 6 --- Relational \& Bitwise Expressions}

\author{J. Zarnett\\
\texttt{jzarnett@uwaterloo.ca}}
\institute{Department of Electrical and Computer Engineering \\
  University of Waterloo}
\date{\today}

\begin{document}

\begin{frame}
  \titlepage
  
  \begin{center}
  \small{Acknowledgments: W.D. Bishop}
  \end{center}
 \end{frame}
 
\begin{frame}
\frametitle{Boolean Variables \& Relational Operators}
Recall from earlier that we introduced the boolean variable concept.\\
\quad The boolean variable can only contain \texttt{true} or \texttt{false}.

Using an arithmetic operator on two numbers, we get a numeric result; using a \alert{relational operator}, we get a boolean result.

Relational operators \alert{compare} two operands and return \texttt{true} or \texttt{false}.

Example: the result of $7 > 5$ is \texttt{true}.

\end{frame}

\begin{frame}
\frametitle{Relational Operators}

In C++, the relational operators are:

\begin{center}
\begin{tabular}{l|l}
\textbf{Operator} & \textbf{Definition} \\ \hline
	\texttt{==} & Equal to\\ \hline
	\texttt{!=} & Not equal to\\ \hline
	\texttt{\textless} & Less than \\ \hline
	\texttt{\textgreater} & Greater than \\ \hline
	\texttt{\textless=} & Less than or equal to \\ \hline
	\texttt{\textgreater=} & Greater than or equal to\\ 
\end{tabular}
\end{center}

Use of \texttt{\textless} where \texttt{\textless=} was intended, and vice versa, is a common source of error. Same for \texttt{\textgreater} and \texttt{\textgreater=}.

\end{frame}

\begin{frame}
\frametitle{Relational Operator Examples}
A few relational operator examples:

\texttt{500 == 401} $\rightarrow$ \texttt{false}\\
\texttt{450.0 <= 450.0} $\rightarrow$ \texttt{true}\\
\texttt{7 != 0} $\rightarrow$ \texttt{true}\\
\texttt{7 != 7} $\rightarrow$ \texttt{false}

Like arithmetic expressions, variables can be in relational expressions.

\texttt{x > 0} $\rightarrow$ ? The result depends on the current value stored in \texttt{x}.

\end{frame}

\begin{frame}
\frametitle{Comparing Numbers}

Use caution when comparing numbers using \texttt{float}, \texttt{double}, and \texttt{decimal} types.

The \texttt{==} and \texttt{!=} operators especially may give results that appear erroneous when using real values.

For example, $7.2000001$ and $7.2$ are not equivalent.

Truncation, rounding, and the internal representation in memory may result in numbers that differ slightly.

\end{frame}


\begin{frame}
\frametitle{Comparing Numbers}

How does this happen? Think of $\frac{1}{3}$.

The decimal (non-fractional) representation is: $0.33333333333...$\\
\quad It takes an infinite number of digits to represent this correctly.

Multiply the decimal representation by 3, the result is $0.99999999...$\\
\quad To the computer, this is not the same as $1$.

This also applies in the computer's representation of numbers. 

Like $\frac{1}{3}$ for us in decimal form, to accurately store \texttt{0.001d} in memory, it would take an infinite number of bits.

\end{frame}

\begin{frame}
\frametitle{Comparing Numbers Solution}

When comparing two real values for equality, a \alert{tolerance} may help.

Instead of comparing with \texttt{==} or \texttt{!=}, compute the difference.

The absolute value of the difference must be less than a set tolerance.

The difference of $7.2000001$ and $7.2$ is $0.0000001$.\\
\quad Then compare $0.0000001$ to the tolerance (for example, $0.01$).

If the difference is less than the tolerance, consider the numbers equivalent.

\end{frame}

\begin{frame}
\frametitle{Boolean Logic}
So far: comparison operators are where we compare two numbers.

What about comparing boolean values to each other?

There is another set of operators for this: \alert{logical operators}.

This follows the rules of \alert{boolean algebra}, which we'll examine now.

\end{frame}


\begin{frame}
\frametitle{Boolean Algebra}
Like the \texttt{bool} variable, the basic values are \texttt{true} and \texttt{false}.

In many mathematical texts, false is represented as 0 and true as 1.

There are three basic operations in boolean algebra:

\begin{enumerate}
	\item \alert{AND} (also called conjunction)
	\item \alert{OR} (also called disjunction)
	\item \alert{NOT} (also called negation)
\end{enumerate}

\end{frame}

\begin{frame}
\frametitle{A Note about Notation}

The AND, OR, and NOT operations may appear in mathematical expressions in your future courses such as ECE~103 (Discrete Math).

You may have had some experience with these already. 

\begin{center}
\begin{tabular}{l|c|l}
	\textbf{Operation} & \textbf{Math Notation} & \textbf{C++ Notation} \\ \hline
	AND & $\wedge$ & \texttt{\&\&} \\ \hline
	OR  & $\vee$ & \texttt{||} \\ \hline
	NOT & $\neg$ & \texttt{!} \\
\end{tabular}
\end{center}

In this course, we'll use the C++ notation. 

\end{frame}


\begin{frame}
\frametitle{Boolean Algebra: AND operation}

The result of the AND operation is true only if both $x$ and $y$ are true.

Truth table for the AND operation:

\begin{center}
\begin{tabular}{l|l|l}
	\textbf{x} & \textbf{y} & \textbf{x \&\& y}\\ \hline
	false & false & false \\ \hline
	false & true & false \\ \hline
	true & false & false \\ \hline
	true & true & true \\ 
\end{tabular}
\end{center}

\end{frame}

\begin{frame}
\frametitle{Boolean Algebra: OR operation}


The result of the OR operation is true if $x$ is true or if $y$ is true, or if both are true.

Truth table for the OR operation:

\begin{center}
\begin{tabular}{l|l|l}
	\textbf{x} & \textbf{y} & \textbf{x || y}\\ \hline
	false & false & false \\ \hline
	false & true & true \\ \hline
	true & false & true \\ \hline
	true & true & true \\ 
\end{tabular}
\end{center}

\end{frame}


\begin{frame}
\frametitle{Boolean Algebra: NOT operation}

NOT x is a unary operator.

The result of the NOT operation is true if $x$ is false. It flips the value.

\begin{center}
\begin{tabular}{l|l}
	\textbf{x} & \textbf{!x}\\ \hline
	false & true \\ \hline
	true & false \\ 
\end{tabular}
\end{center}

\end{frame}


\begin{frame}
\frametitle{C++ Boolean Examples}


\texttt{bool a = b \&\& c;}\\
\texttt{bool d = e || f;}\\
\texttt{bool g = !h;}

\end{frame}

\begin{frame}
\frametitle{Short-Circuit Evaluation}
In C++, the AND and OR operations use \alert{short-circuit} behaviour.

What is short-circuit behaviour? When evaluating an expression, we might know the outcome partway through.

Example: $x$ \&\& $y$. If we know $x$ is false, there's no need to look at $y$.\\
\quad Regardless of the value of $y$, if $x$ is false, the result is false.

Code example: \texttt{ x > 5 \&\& y < 7 }\\
If \texttt{x} is less than or equal to 5, the computer won't bother evaluating if \texttt{y} is less than 7 because the outcome is false regardless.

This is an optimization: the computer doesn't have to do the work of retrieving \texttt{y} from memory and comparing it to 7.

\end{frame}

\begin{frame}
\frametitle{Short-Circuit Evaluation}

Short-circuit evaluation also applies to the OR operator.

Further example: \texttt{ a > 1 || b < 0 }\\
If \texttt{a} is greater than 1, the computer won't bother evaluating if \texttt{b} is less than 0 because the outcome is true regardless.

\end{frame}

\begin{frame}
\frametitle{Comparison Operator Precedence}

It's important to note that in C++, the \texttt{!} negation operator has higher precedence than the \texttt{\&\&} and \texttt{||} operators.

Example:
\texttt{bool x = !y \&\& z;} is evaluated as:
\texttt{(!y) \&\& z}\\
\quad and it is NOT evaluated as \texttt{!(y \&\& z);}

As always, precedence can be specified with the use of brackets.

\end{frame}


\begin{frame}
\frametitle{Bitwise Operators}

Bitwise operators are uncommonly used outside of some special cases (like hardware interaction).

These operate on the bit representation of a number in memory.

\end{frame}

\begin{frame}
\frametitle{Bitwise Operator Example}

Let's consider an example using Bitwise AND of some numbers: 248 and 63.

248 in binary is: 1111 1000;  63 in binary is: 0011 1111.

Line them up, and compare each of the individual bits of the first number to the one immediately below.

1111 1000\\
0011 1111

Here we are doing a logical AND operation, and the result is:\\
0011 1000 (56 in decimal).


\end{frame}

\begin{frame}
\frametitle{Bitwise Operators}

A list of the Bitwise Operators of C++:

\begin{center}
\begin{tabular}{l|l}
\textbf{Operator} & \textbf{Definition} \\ \hline
	\texttt{\&} & Bitwise AND\\ \hline
	\texttt{|} & Bitwise OR\\ \hline
	\texttt{\^} & Bitwise XOR\\ \hline
	\texttt{\~} & Bitwise Complement \\ \hline
	\texttt{\textless\textless} & Shift first operand left by number of bits in the second\\ \hline
	\texttt{\textgreater\textgreater} & Shift first operand right by number of bits in the second\\ 
\end{tabular}
\end{center}

Some of these require a bit more explanation...

\end{frame}

\begin{frame}
\frametitle{Bitwise XOR}
XOR (Exclusive OR) is another operation of boolean logic, but it's not a basic one (it's derived from the three we've already seen).

(Its math symbol is $\oplus$)

The result is true if $x$ and $y$ have different values.

Truth table for the XOR operation:

\begin{center}
\begin{tabular}{l|l|l}
	\textbf{x} & \textbf{y} & \textbf{x \^{} y}\\ \hline
	false & false & false \\ \hline
	false & true & true \\ \hline
	true & false & true \\ \hline
	true & true & false \\ 
\end{tabular}
\end{center}

\end{frame}

\begin{frame}
\frametitle{Bitwise Complement}

The bitwise complement is like the negation (\texttt{!}) operation, but at a bitwise level.

1111 1000 (248)\\
becomes:\\
0000 0111 (7)\\

\end{frame}

\begin{frame}
\frametitle{Shift Left, Shift Right}
The left shift operator updates the first operand based on the second.

If the variable \texttt{x} is 8, its bit representation is:\\
0000 1000

\texttt{int y = x << 1;}\\
Shifts the bits left by 1.

Then \texttt{y} is 16; its bit representation is:\\
0001 0000

Right shift works the same way as the left shift operator works, but in the other direction.

\end{frame}

\begin{frame}
\frametitle{Relational Operators \& Type Promotion}
Relational and bitwise operators still follow the type promotion rules.

The compiler will convert one of the operands, if necessary.

\end{frame}



\begin{frame}
\frametitle{All Assignment Operators}

Now we can see the full list of the assignment operators of C++:

\begin{center}
\begin{tabular}{l|l}
\textbf{Operator} & \textbf{Definition} \\ \hline
	\texttt{=} & Assigns the result to the variable\\ \hline
	\texttt{+=} & Adds the result to the variable\\ \hline
	\texttt{-=} & Subtracts the result from the variable\\ \hline
	\texttt{*=} & Multiplies the variable by the result\\ \hline
	\texttt{/=} & Divides the variable by the result\\ \hline
	\texttt{\%=} & Assigns the remainder to the variable\\ \hline
	\texttt{\&=} & Assigns the bitwise AND to the variable\\ \hline
	\texttt{|=} & Assigns the bitwise OR to the variable\\ \hline
	\texttt{\^{}=} & Assigns the complement to the variable\\ \hline
	\texttt{\textless\textless =} & Shifts the variable left by result bits\\ \hline
	\texttt{\textgreater\textgreater =} & Shifts the variable right by result bits\\
\end{tabular}
\end{center}
\end{frame}

\end{document}

